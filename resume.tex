%%%%%%%%%%%%%%%%%
% This is an sample CV template created using altacv.cls
% (v1.3, 10 May 2020) written by LianTze Lim (liantze@gmail.com). Now compiles with pdfLaTeX, XeLaTeX and LuaLaTeX.
%
%% It may be distributed and/or modified under the
%% conditions of the LaTeX Project Public License, either version 1.3
%% of this license or (at your option) any later version.
%% The latest version of this license is in
%%    http://www.latex-project.org/lppl.txt
%% and version 1.3 or later is part of all distributions of LaTeX
%% version 2003/12/01 or later.
%%%%%%%%%%%%%%%%

%% If you are using \orcid or academicons
%% icons, make sure you have the academicons
%% option here, and compile with XeLaTeX
%% or LuaLaTeX.
% \documentclass[10pt,a4paper,academicons]{altacv}

%% Use the "normalphoto" option if you want a normal photo instead of cropped to a circle
% \documentclass[10pt,a4paper,normalphoto]{altacv}

\documentclass[10pt,a4paper,ragged2e,withhyper]{altacv}
%% AltaCV uses the fontawesome5 and academicons fonts
%% and packages.
%% See http://texdoc.net/pkg/fontawesome5 and http://texdoc.net/pkg/academicons for full list of symbols. You MUST compile with XeLaTeX or LuaLaTeX if you want to use academicons.

% Change the page layout if you need to
\geometry{left=1.25cm,right=1.25cm,top=1.5cm,bottom=1.5cm,columnsep=1.2cm}

% The paracol package lets you typeset columns of text in parallel
\usepackage{paracol}

% Change the font if you want to, depending on whether
% you're using pdflatex or xelatex/lualatex
\ifxetexorluatex
  % If using xelatex or lualatex:
  \setmainfont{Roboto Slab}
  \setsansfont{Lato}
  \renewcommand{\familydefault}{\sfdefault}
\else
  % If using pdflatex:
  \usepackage[rm]{roboto}
  \usepackage[defaultsans]{lato}
  % \usepackage{sourcesanspro}
  \renewcommand{\familydefault}{\sfdefault}
\fi

% Change the colours if you want to
\definecolor{SlateGrey}{HTML}{2E2E2E}
\definecolor{LightGrey}{HTML}{666666}
\definecolor{DarkPastelRed}{HTML}{450808}
\definecolor{PastelRed}{HTML}{8F0D0D}
\definecolor{GoldenEarth}{HTML}{E7D192}
\colorlet{name}{black}
\colorlet{tagline}{PastelRed}
\colorlet{heading}{DarkPastelRed}
\colorlet{headingrule}{GoldenEarth}
\colorlet{subheading}{PastelRed}
\colorlet{accent}{PastelRed}
\colorlet{emphasis}{SlateGrey}
\colorlet{body}{LightGrey}

% Change some fonts, if necessary
\renewcommand{\namefont}{\Huge\rmfamily\bfseries}
\renewcommand{\personalinfofont}{\footnotesize}
\renewcommand{\cvsectionfont}{\LARGE\rmfamily\bfseries}
\renewcommand{\cvsubsectionfont}{\large\bfseries}


% Change the bullets for itemize and rating marker
% for \cvskill if you want to
\renewcommand{\itemmarker}{{\small\textbullet}}
\renewcommand{\ratingmarker}{\faCircle}

%% sample.bib contains your publications
\addbibresource{sample.bib}

\begin{document}
\name{Yuji Tabata}
\tagline{Software Engineer}
%% You can add multiple photos on the left or right
% \photoR{2.8cm}{Globe_High}
% \photoL{2.5cm}{Yacht_High,Suitcase_High}

\personalinfo{%
  % Not all of these are required!
  \email{yuji.jd.tabata@gmail.com}
  \location{2291-2, Hamagawa, Takasaki, Gunma, 3700081, JAPAN}
  \linkedin{yuji-tabata}
  %% You MUST add the academicons option to \documentclass, then compile with LuaLaTeX or XeLaTeX, if you want to use \orcid or other academicons commands.
  % \orcid{0000-0000-0000-0000}
  %% You can add your own arbtrary detail with
  %% \printinfo{symbol}{detail}[optional hyperlink prefix]
  % \printinfo{\faPaw}{Hey ho!}[https://example.com/]
  %% Or you can declare your own field with
  %% \NewInfoFiled{fieldname}{symbol}[optional hyperlink prefix] and use it:
  % \NewInfoField{gitlab}{\faGitlab}[https://gitlab.com/]
  % \gitlab{your_id}
}

\makecvheader
%% Depending on your tastes, you may want to make fonts of itemize environments slightly smaller
% \AtBeginEnvironment{itemize}{\small}

%% Set the left/right column width ratio to 6:4.
\columnratio{0.6}

% Start a 2-column paracol. Both the left and right columns will automatically
% break across pages if things get too long.
\begin{paracol}{2}
\cvsection{Experience}

\cvevent{Software Engineer}{SoftBank Corp.}{Month 2018 -- Ongoing}{Tokyo, JAPAN}
\begin{itemize}
\item 
  Invented an NFC tag application framework targeted for big companies.
  I am leading a group of 10 people to introduce this framework into the SoftBank's corporate iOS app to improve the accessibility for the employees to the facilities in the office.

\item
  Migrated GitLab, Kanboard (a Kanban tool), and other server software from on-premises servers to Google Cloud Platform.
  Adopted Kubernetes to maintain the infrastructure as code.

\item
  Created a check-in system for SoftBank's company-owned buses to analyze their usage pattern.
  Employees scan their employee badge to the Android device situated at the entrance of the bus.
  This system had been operated for six months with 100 to 150 users per day on average.

\item 
  Developed a security software for employees of SoftBank, which monitors and sends the information about currently active windows on the Windows PCs.
  I was responsible for building the Windows application.

\item
  Created a web dashboard for smart CCTVs.
  I made a heatmap component to overlay on the original footage to make congestions apparent. 

\item
  Demonstrated a cutting-edge web conferencing system which SoftBank had developed at a technology tradeshow in Cambridge, England.  

\end{itemize}

\divider

\cvevent{Intern}{Cyber Defence Institute, Inc.}{June 2017}{Tokyo, JAPAN}

\begin{itemize}
  \item Leaned practical computer security and wrote a fuzzing tool for MQTT servers to find a security hole.
\end{itemize}

\cvsection{Education}

\cvevent{M.S.\ in Information Science}{Tohoku University}{Apr 2016 -- Mar 2018}{Sendai, JAPAN}

\begin{itemize}
  \item Researched the safety of messaging protocols from the perspective of theoretical cryptography.
  \item Studied the mathematical foundation of modern cryptography.
\end{itemize}

\divider

\cvevent{B.E.\ in Computer Science}{Tohoku University}{Apr 2012 -- Mar 2016}{Sendai, JAPAN}
\begin{itemize}
  \item Researched the methodology to prove the semantics of functional programs.
  \item Studied the mathematical foundation of functional programming languages.
\end{itemize}

\medskip

% \cvsection{A Day of My Life}

% % Adapted from @Jake's answer from http://tex.stackexchange.com/a/82729/226
% % \wheelchart{outer radius}{inner radius}{
% % comma-separated list of value/text width/color/detail}
% \wheelchart{1.0cm}{0.0cm}{%
%   6/8em/accent!30/{Sleep,\\beautiful sleep},
%   3/8em/accent!40/Hopeful novelist by night,
%   8/8em/accent!60/Daytime job,
%   2/10em/accent/Sports and relaxation,
%   5/6em/accent!20/Spending time with family
% }

%% Switch to the right column. This will now automatically move to the second
%% page if the content is too long.
\switchcolumn


%% Yeah I didn't spend too much time making all the
%% spacing consistent... sorry. Use \smallskip, \medskip,
%% \bigskip, \vpsace etc to make ajustments.

% \divider

\cvsection{Message}

I am a software engineer with a lot of 0 to 1 development experience.
I am looking for an exciting job opportunity in Singapore.

% \medskip

\cvsection{Side Projects}

\cvachievement{\faCarSide}{Bluetooth-Controlled Car}{Control a toy car with your iPhone}

\divider

\cvachievement{\faTablets}{NFC Candy Dispenser Machine}{Scan an NFC tag to get candies}

\divider

\cvachievement{\faHeartbeat}{Mood Tracking App}{Monitor your emotional cycle}

\cvsection{Personal history}

\cvachievement{\faTrophy}{Runner-up Best Attraction}{Showcased my realtime strategy game at my high school's annual festival.}

\divider

\cvachievement{\faHiking}{Survival in India for 2 Weeks}{Robbed of everything on day 1 except for my passport and a flight ticket.}

\divider

\cvachievement{\faTheaterMasks}{Shakespeare Company Japan}{Performed as an actor in many places including Tokyo and tsunami-hit areas.}

%\divider

%\cvachievement{\faGifts}{Charity Santa}{Visited families with difficulties and handed out gifts in disguise of Santa Claus.}

\cvsection{Skills}

\cvtag{Project Leadership}
\cvtag{System Design}

\divider

\cvtag{GCP}
% \cvtag{Azure}
% \cvtag{AWS}
\cvtag{Firebase}
\cvtag{Docker}
\cvtag{Kubernetes}
\divider

\cvtag{Web}
\cvtag{iOS}
\cvtag{Android}
\cvtag{Arduino}
\cvtag{Linux}
\cvtag{macOS}
\cvtag{Windows}

\divider

\cvtag{Spring}
\cvtag{Django}
\cvtag{Express}
\cvtag{React}

\divider

\cvtag{Swift}
\cvtag{Kotlin}
\cvtag{Java}
\cvtag{Go}
\cvtag{Python}
\cvtag{TypeScript}
\cvtag{JavaScript}
\cvtag{C}
\cvtag{C{\#}}
\cvtag{Rust}
\cvtag{Shell}
\cvtag{Lisp}
\cvtag{Haskell}

\cvsection{Languages}

\cvskill{Japanese}{5}
\divider
\cvskill{English}{4}
\divider
\cvskill{Mandarin}{1}

\end{paracol}


\end{document}
